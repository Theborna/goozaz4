\section{سوال 2) آشنایی با مقادیر عناصر}
\subsection*{\textbf{الف)}}
ده عدد مقاومت داریم با مقادیر $R_N = 1k\Omega$ و
10 عدد مقاومت با مقادیر $C_N = 10\mu F$ را اندازه
میگیریم.\\
برای اندازه گیری دقیق تر مولتی متر را روی پایه مناسب قرار میدهیم($2k\Omega$)
و برای خازن ها در پایه میکرو فاراد با رعایت پلاریتی.
\begin{latin}
    \begin{table}[h]
        \centering
        \begin{tabular}{|c|c|c|c|c|c|c|c|c|c|c|}
            \hline
            N                       & 1      & 2      & 3      & 4      & 5      & 6      & 7      & 8      & 9      & 10     \\
            \hline
            $R_N(k\Omega)$          & 0.9814 & 1.0044 & 0.9945 & 0.9948 & 0.9965 & 0.9884 & 0.9911 & 1.0186 & 0.9993 & 0.9853 \\
            \hline
            $C_{polarized}(\mu F)$  & 9.04   & 9.15   & 8.00   & 8.94   & 8.87   & 8.82   & 9.11   & 8.35   & 8.77   & 8.75   \\
            \hline
            $C_{non\ polar}(\mu F)$ & 9.88   & 9.97   & 9.05   & 9.59   & 9.16   & 9.79   & 9.70   & 9.06   & 9.27   & 9.59   \\
            \hline
        \end{tabular}
        \caption{measured values}
        % \label{\ref{table:RC}}
    \end{table}
\end{latin}
قابل مشاهده است که برای هیچ کدام از مقادیر اندازه گیری شده
هیچ یک از مقادیر برابر نبوده و خطای داریم.\\
حال این خطا را تحلیل میکنیم.
\subsubsection{تحلیل خطا}
روابط استفاده شده:$$\bar{x}=\frac{\Sigma_{i=0}^N\ x_i}{N}\ ,
    \ \sigma = \sqrt{\frac{\Sigma_{i=0}^N\ (x_i - \bar{x})^2}{N-1}}$$
برای مقاومت ها:\\
$$\bar{R} = 1.00k\Omega\ , \ \sigma_R = 0.02k\Omega,R_{max}=1.0186k\Omega,R_{min}=0.9814k\Omega$$
برای خازن های پولار:\\
$$\bar{C} = 8.8\mu F\ , \ \sigma_C = 0.3\mu F,C_{max}=9.15\mu F,C_{min}=8.00\mu F$$
برای خازن های غیر پولار:\\
$$\bar{C} = 9.5\mu F\ , \ \sigma_C = 0.3\mu F,C_{max}=9.88\mu F,C_{min}=9.05\mu F$$
\pagebreak